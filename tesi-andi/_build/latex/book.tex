%% Generated by Sphinx.
\def\sphinxdocclass{jupyterBook}
\documentclass[letterpaper,10pt,english]{jupyterBook}
\ifdefined\pdfpxdimen
   \let\sphinxpxdimen\pdfpxdimen\else\newdimen\sphinxpxdimen
\fi \sphinxpxdimen=.75bp\relax
\ifdefined\pdfimageresolution
    \pdfimageresolution= \numexpr \dimexpr1in\relax/\sphinxpxdimen\relax
\fi
%% let collapsible pdf bookmarks panel have high depth per default
\PassOptionsToPackage{bookmarksdepth=5}{hyperref}
%% turn off hyperref patch of \index as sphinx.xdy xindy module takes care of
%% suitable \hyperpage mark-up, working around hyperref-xindy incompatibility
\PassOptionsToPackage{hyperindex=false}{hyperref}
%% memoir class requires extra handling
\makeatletter\@ifclassloaded{memoir}
{\ifdefined\memhyperindexfalse\memhyperindexfalse\fi}{}\makeatother

\PassOptionsToPackage{warn}{textcomp}

\catcode`^^^^00a0\active\protected\def^^^^00a0{\leavevmode\nobreak\ }
\usepackage{cmap}
\usepackage{fontspec}
\defaultfontfeatures[\rmfamily,\sffamily,\ttfamily]{}
\usepackage{amsmath,amssymb,amstext}
\usepackage{polyglossia}
\setmainlanguage{english}



\setmainfont{FreeSerif}[
  Extension      = .otf,
  UprightFont    = *,
  ItalicFont     = *Italic,
  BoldFont       = *Bold,
  BoldItalicFont = *BoldItalic
]
\setsansfont{FreeSans}[
  Extension      = .otf,
  UprightFont    = *,
  ItalicFont     = *Oblique,
  BoldFont       = *Bold,
  BoldItalicFont = *BoldOblique,
]
\setmonofont{FreeMono}[
  Extension      = .otf,
  UprightFont    = *,
  ItalicFont     = *Oblique,
  BoldFont       = *Bold,
  BoldItalicFont = *BoldOblique,
]



\usepackage[Bjarne]{fncychap}
\usepackage[,numfigreset=1,mathnumfig]{sphinx}

\fvset{fontsize=\small}
\usepackage{geometry}


% Include hyperref last.
\usepackage{hyperref}
% Fix anchor placement for figures with captions.
\usepackage{hypcap}% it must be loaded after hyperref.
% Set up styles of URL: it should be placed after hyperref.
\urlstyle{same}

\addto\captionsenglish{\renewcommand{\contentsname}{Introduzione e motivazioni}}

\usepackage{sphinxmessages}



        % Start of preamble defined in sphinx-jupyterbook-latex %
         \usepackage[Latin,Greek]{ucharclasses}
        \usepackage{unicode-math}
        % fixing title of the toc
        \addto\captionsenglish{\renewcommand{\contentsname}{Contents}}
        \hypersetup{
            pdfencoding=auto,
            psdextra
        }
        % End of preamble defined in sphinx-jupyterbook-latex %
        

\title{Tesi Andi Dulla}
\date{Apr 27, 2022}
\release{}
\author{Andi Dulla}
\newcommand{\sphinxlogo}{\vbox{}}
\renewcommand{\releasename}{}
\makeindex
\begin{document}

\pagestyle{empty}
\sphinxmaketitle
\pagestyle{plain}
\sphinxtableofcontents
\pagestyle{normal}
\phantomsection\label{\detokenize{intro::doc}}


\sphinxAtStartPar
Le reti di sensori wireless, comunemente chiamate WSN Wireless Sensor Networks, vengono usate per molte applicazioni a
lungo termine, come ad esempio:
\begin{itemize}
\item {} 
\sphinxAtStartPar
Applicazioni militari

\item {} 
\sphinxAtStartPar
Città intelligenti e servizi alla popolazione

\item {} 
\sphinxAtStartPar
Industria 4.0 e smart factories

\item {} 
\sphinxAtStartPar
veicoli intelligenti

\end{itemize}

\sphinxAtStartPar
Il collante che tiene tutto insieme è il trasporto di informazioni, la cosa che accomuna tutte le applicazioni elencate
come esempio

\sphinxstepscope


\part{Introduzione e motivazioni}

\sphinxstepscope


\chapter{Reti di sensori}
\label{\detokenize{chapters/sensor:reti-di-sensori}}\label{\detokenize{chapters/sensor::doc}}
\sphinxAtStartPar
Una rete di sensori solitamente consiste in device:
\begin{itemize}
\item {} 
\sphinxAtStartPar
sorgenti, i sensori che campionano l’ambiente e producono informazioni utili

\item {} 
\sphinxAtStartPar
terminazioni, le base station o i clusterhead

\item {} 
\sphinxAtStartPar
usati dagli utenti finali

\end{itemize}

\sphinxAtStartPar
I nodi sensori vengono installati in zone geografiche d’interesse, dove grazie alle loro caratteristiche di costo
ridotto, auto\sphinxhyphen{}organizzazione e intercambiabilità permettono di avere un piano di raccolta dati fungibile e capillare.
Per poter raggiungere l’obiettivo di avere una rete auto\sphinxhyphen{}organizzante per monitorare ambienti ho bisogno di campionare
il fenomeno d’interesse, codificarlo e trasmettere questa rappresentazione verso un nodo consumatore dell’informazione
attraverso un protocollo di routing.

\sphinxAtStartPar
Un’ipotesi che spesso viene fatta è che i nodi consumatori siano più potenti dei nodi campionatori. In figura vediamo
uno scenario tipo di rete di sensori. È da osservare la presenza della nuvoletta che rappresenta la Wide Area Network
che metterà in comunicazione il data center di elaborazione dati e l’utente finale.

\begin{sphinxShadowBox}
\sphinxstylesidebartitle{Clustertree}

\sphinxAtStartPar
Struttura a grafo aciclico in cui ogni nodo è un cluster di nodi rappresentati da un super nodo detto clusterhead
\end{sphinxShadowBox}

\begin{figure}[htbp]
\centering
\capstart

\noindent\sphinxincludegraphics[width=700\sphinxpxdimen]{{WSN_topo}.jpg}
\caption{Scenario tipo organizzato come \sphinxstylestrong{cluster\sphinxhyphen{}tree}!}\label{\detokenize{chapters/sensor:markdown-fig}}\end{figure}

\sphinxAtStartPar
Com’è composto il singolo device sensore? A grandi linee possiamo pensarlo come composto dai seguenti blocchi in figura:

\begin{sphinxShadowBox}
\sphinxstylesidebartitle{Precisazione sullo schema a blocchi}

\sphinxAtStartPar
I blocchi tratteggiati sono opzionali
\end{sphinxShadowBox}

\begin{figure}[htbp]
\centering
\capstart

\noindent\sphinxincludegraphics[width=700\sphinxpxdimen]{{eSAKY8c}.png}
\caption{Schema a blocchi di com’è composto un sensore intelligente}\label{\detokenize{chapters/sensor:id1}}\end{figure}

\sphinxAtStartPar
Le reti di sensori sono molto particolari per i seguenti motivi:
\begin{enumerate}
\sphinxsetlistlabels{\arabic}{enumi}{enumii}{}{.}%
\item {} 
\sphinxAtStartPar
I nodi sensori sono limitati in termini di batteria (energia), capacità computazionale e memoria. Il parametro più
sensibile è proprio la durata della batteria. Data la natura wireless delle comunicazioni bisogna fare un bilancio di
tratta e ricordarsi che la \sphinxstyleemphasis{path loss} è il fattore predominante nel dispendio necessario per trasmettere da A a B.
\( FSPL = P_T G_T G_R (\frac{λ}{4πd})^2 \)  dunque vediamo come la riduzione della potenza al ricevitore segue una legge
quadratica \( d^{-2} \). Questo è un problema che induce a progettare bene gli algoritmi di routing, in quanto una trasmissione
completamente 1\sphinxhyphen{}hop è impensabile, servono algoritmi di routing multi\sphinxhyphen{}hop con le loro peculiarità.

\item {} 
\sphinxAtStartPar
Dato il costo ridotto dei dispositivi sensori è possibile installare un gran numero di dispositivi. Avere un gran numero
di dispositivi significa coprire meglio il territorio ed essere più capillari (e di conseguenza efficaci) nel monitoraggio.
D’altro canto non tutti i dispositivi verrebbero usati, avrei una certa percentuale inutilizzata e questo riduce l’efficienza
della rete. Come in molti altri casi la questione sta nel trovare un equilibrio accettabile.

\end{enumerate}

\begin{sphinxShadowBox}
\sphinxstylesidebartitle{Grado di connessione}

\sphinxAtStartPar
È la distribuzione di porbabilità del numero di collegamenti logici che un dispositivo ha con i suoi vicini
\end{sphinxShadowBox}
\begin{enumerate}
\sphinxsetlistlabels{\arabic}{enumi}{enumii}{}{.}%
\item {} 
\sphinxAtStartPar
Dopo il rollout, i nodi si organizzano in una rete e cooperano per portare a termine un compito. In generale non c’è
un nodo coordinatore centrale che orchestra tutto quanto. Inoltre data la natura della rete dopo verrà dimostato come il
fallimento o guasto casuale nelle reti di sensori che soddisfano un criterio di normalità sulla distribuzione dei gradi
di connessione, non vengono influenzati in modo rilevante dai guasti random.

\item {} 
\sphinxAtStartPar
La topologia della rete cambia di frequente poiché oiltre ai guasti dobbiamo tenere conto anche dei cicli di
sleep/operazione del dispositivo, oltre che al cambio di ruolo che ogni nodo ricoprirà (vedi capitoli successivi).

\end{enumerate}

\sphinxstepscope


\part{SDN e Controller}

\sphinxstepscope

\sphinxstepscope


\part{Algoritmi di ottimizzazione}

\sphinxstepscope


\chapter{}
\label{\detokenize{chapters/optmization:id1}}\label{\detokenize{chapters/optmization::doc}}






\renewcommand{\indexname}{Index}
\printindex
\end{document}